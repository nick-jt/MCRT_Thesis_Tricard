\addchapheadtotoc
\chapter{Conclusions and future work}~\label{chapter:conclusion}

\section{Summary of findings}
The influences of thermal radiation are known to be significant in combustion systems. A brief overview of the fundamentals of radiation, as well as categorized effects of radiation in combustion specifically, were discussed in Chapter~\ref{chapter:Importance}. 
Radiation has been shown to contribute significantly to changes in the fluid dynamics, boundary interactions, and chemical characteristics in and around the flame. Radiation acts as a diffusive process of thermal energy, where high temperature and opaque regions emit strongly to the medium and surroundings. 
This usually introduces reductions in peak flame temperatures that may stymie combustion, slow buoyant forces, and reduce turbulence in the flame. Increases in peak temperatures have also been observed due to irradiation of particles elsewhere in the medium improving combustion efficiency.
Overall, radiative source and sink terms in the gas tend to be less than heat release from combustion reactions. As a result, the contribution of radiation is more meaningful across long timescales compared to the fast chemistry. Therefore, an understanding of slow processes like chemical production of pollutants NOx and soot require accurate modeling of radiation.
In power-generating combustion devices, the exact role of radiation remains poorly understood. Previous studies suggest radiative emission from aeronautical gas-turbines can contribute an identical order of magnitude heat flux to the boundaries as convective heat transfer. Some studies have demonstrated peak temperature reductions of up to 200K when radiation is introduced, while others have shown temperature increases resulting from improved combustion in highly-absorbing regions. Overall analysis shows as much as 80\% of the radiative emission in fuel-rich zones is from soot. Rocket engines exhibit between 5\% and 30\% of overall heat flux to the walls resulting from radiative emission, largely located near the nozzle throat where temperatures are the highest.
Radiation is also significant in wild-fires and other hazardous combustion scenarios. Heat flux can propagate a large distance from the flame and ignite nearby objects, leading to fire-spread.

The background of the Monte-Carlo method of modeling radiation is next discussed in Chapter~\ref{chapter:Modeling}. The Monte-Carlo method is broadly known to be the most accurate, robust, but also one of the most computationally expensive radiation models used.
However, with computational power increasing, it is broadly recognized to be used more frequently in the future. First, an account of the fundamentals of MCRT, including the random number relations of ray point of emission, direction and wavenumber for forward Monte-Carlo ray tracing is discussed.
A description of the tracing procedure through the mesh, and to the boundaries is provided.
Several alternative mathematical techniques, as well as re-formulations of the Monte-Carlo ray tracing method are presented, along with their various advantages. Next, various influences of ray-tracing from the field of computer graphics are discussed, including the Graphics Processing Unit (GPU) and Bounding Volume Hierarchy (BVH).
Previous literature is reviewed on applications of these methods, and it is found that the relatively few researchers that have applied GPUs to MCRT for thermal radiation modeling observed tremendous runtime reductions, and that no other researchers had applied the BVH to the calculation of a participating medium.
Finally, a description of the radiation implementation for this research is presented, where both GPUs and BVHs are applied, as well as the highest fidelity non-gray model, line-by-line, and a robust tracing procedure through an unstructured polyhedral mesh.
The solver is integrated with the \verb|OpenFOAM| CFD software package, which enables its use with coupled, time-accurate combustion simulations through a source term contribution to the energy equation. Additionally, the Kokkos programming model and ArborX geometric search library are used for interchangeable GPU/CPU compatibility and BVH implementation, respectively. Three primary ray tracing implementations are presented: Standard-Forward, ArborX-Reverse, and Hybrid-Reverse. Each offers a different approach to ray tracing, utilizing standard forward ray tracing, BVH-based reverse ray tracing, and a combination of both, respectively. A distributed raytracing procedure, where rays are emitted and traced across multiple MPI ranks simultaneously, is presented for use with both ArborX-Reverse and Hybrid-Reverse to improve performance in high performance computing systems.

The present solvers are applied to four geometries of varying complexity: a plane-parallel participating medium, a vitiated backward-facing step combustor, a detailed turbulent pool-fire, and a time-accurate turbulent pool-fire.
% Plane-parallel
One dimensional plane-parallel results compare excellently against exact solutions for all three primary tested versions. Tests are conducted for a variety of absorption coefficients, medium temperatures, and ray counts.
Radiative absorption increased with absorption coefficient and temperature, and stochastic variability become more apparent with lower ray-counts.
ArborX-Reverse and Hybrid-Reverse tracing methods demonstrate increased variance near the boundaries and for lower absorption coefficients due to greater numbers of reverse-rays exiting the domain.
The solvers each require approximately the same runtime for all configurations, and results compare equally well when run with eight MPI ranks.
% Backward Facing step
Then, a 3-D backward-facing step combustor is simulated at a moderate temperature of approximately $800$K. Results are compared against those of an established Fortran-radiation model. Excellent agreement is obtained. It is found that a runtime reduction of $60$\% can be seen between the established Fortran implementation compared to the newer MCRT, and GPU parallel processing further reduces runtimes by approximately $25$\%. The speedups are a result of two advancements: improvements in parallel implementation through the introduction of the Kokkos back-end, and improved memory management through minimal duplication of mesh-data within the CPU.
% DNS POOL FIRE:
Next, the 3-D direct numerical simulation of a pool-fire flame is used to demonstrate the use of the new MCRT model. Runtimes for various parallel back-ends including GPU, CPU, and serial are compared. Both NVIDIA V100 and NVIDIA A100 GPUs are shown to accelerate the solver over two orders of magnitude, with the A100s showing almost 400 times speedup. OpenMP parallel execution also shows an order of magnitude speedup over serial. Next, radiation is simulated again using a uniform-absorption coefficient gray model, and compared amongst the MCRT, fvDOM and P1 solvers. fvDOM is shown to have reasonably good accuracy while P-1 deviates significantly from baseline MCRT results. Meanwhile, the GPU-accelerated MCRT solver is shown to run faster than both serial fvDOM and serial P-1 radiation solutions within a gray medium, indicating its great potential to replace these solvers in time-accurate combustion simulations.
% TRANSIENT POOL FIRE:
Lastly, a similar turbulent pool fire is simulated from its initial conditions in a transient simulation. First, a frozen-field analysis is conducted on an early time-step of the simulation. Results show a strong redistribution of energy throughout the flame and to the surrounding walls. Trends between radiative absorption and emission with the Planck-mean absorption coefficient and flame temperature are outlined. A minimal degree of radiative absorption is seen in the surrounding medium. Additionally, the radiative re-absorption in the flame increased considerably when the line-by-line non-gray model was implemented. Detailed account of runtime consumption in the model is described through the use of a profiling tool. It is found that a significant portion of the runtime is consumed through the loading and transferal of the non-gray database. These results were magnified for the GPU, where several deep copies of CPU to GPU memory must be performed.
Results and performance are then presented considering all timesteps in the run. Interpretability of the physical effects on the pool flame were expected to be more accurate because coupled influences of radiation can only be accounted for by simulating from initial conditions. Results show a significant decrease in temperature of up to $600$ K compared to the simulation without radiation. Profiling results display a consistent $25$\% of the runtime was consumed by radiation for each time-step, on average. Noting the results from the profiling of the frozen-field analysis, it was concluded that the consumption in runtime would decrease if the mesh and non-gray information are maintained in memory between time-steps.

% Finally, succesful validation of the present radiation model encouraged the modeling of a Pratt \& Whitney NEO combustor with line-by-line accurate non-gray modeling. A scatter diagram of various computational parameters in the medium revealed soot exerted a controlling influence on the radiation emission characteristics in the flame. At temperatures beyond a crossover point, soot volume fractions diminished rapidly, resulting in an equally rapid decrease in radiative emission, leading to a local maximum in radiative emission as a function of temperature. The boundaries of the combustor displayed a much higher degree of radiative heat flux adjacent to the fuel-rich region of the combustor. The influence of non-gray modeling was then presented. The line-by-line non-gray model as shown to increase radiative re-absorption significantly and decrease wall-incident radiative heat flux by up to $60$\%. The contributions of various chemical species to the wall-incident heat flux was evaluated, and it was found that soot contributed the most, followed by CO$_2$. 
% Profiling of the radiation model on the combustor showed a reduction of runtime by approximately 75\%, and the non-gray calculation increased runtimes by 10\%.

\section{Recommendations for Future Work}
The present implementation of MCRT with GPUs has shown significant speedup compared to CPU implementations. However, it is anticipated that further speedups can be obtained by implementing improved GPU-favorable memory coalescence and by sorting rays by absorption coefficient~\cite{Silvestri2019ASimulation}.
Furthermore, a fully asynchronous computation of radiation and CFD on the GPU and CPU simultaneously would reduce the apparent runtime of radiation to only the communication time between the two. The potential for the Bounding Volume Hierarchy was limited in the present demonstration, however, the BVH may have potential to provide improvements for a larger distributed-memory simulations.
Null-collision Monte-Carlo has demonstrated significant potential to accelerate MCRT. Combining this method with advanced mesh approaches, such as the BVH, could bring measurable speedups to the ray-tracing calculation.
Improvements such as these, along with those presented in this thesis, will enable scientists and engineers to have more complete understanding of the influence of radiation within their combustion systems. 