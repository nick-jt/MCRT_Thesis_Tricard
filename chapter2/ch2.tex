\addchapheadtotoc
\chapter{Modeling of radiation in combustion systems}\label{chapter:models_and_methods}

\section{Fundamentals of radiation} \label{Sec:FundOfRad}
Beer's law, Kirchoff's law, black body radiation, gray vs non-gray.

% =============================================================== %
\section{Radiative Transfer Equation}
The fundamental equation to describe radiation in an absorbing-emitting medium is the radiative transfer equation (RTE), eq. \ref{eq:RTE2}.

\begin{equation}
    \frac{dI_\eta{}}{d\textbf{s}} = \textbf{s} \cdot \nabla{I_\eta{}} = \kappa{}_\eta{}I_{b\eta{}}-\kappa{}_\eta{}I_\eta{}-\sigma{}_{s\eta{}}I_\eta{}+\frac{\sigma{}_{s\eta{}}}{4\pi}\int_{4\pi{}}{I_\eta{}(\hat{s})\Phi_\eta{}(\hat{s}_i,\hat{s})}\,d\Omega{}_i
    \label{eq:RTE2}
\end{equation}

Where $I$ is ray intensity, $\textbf{s}$ is a path length (with both a direction and a location), $\kappa{}$ is the absorption coefficient, $I_b$ is the black-body intensity, $\sigma{}$ is the scattering coefficient, and $\Phi{}$ is the scattering phase-function (representing the probability a ray from direction $\hat{s}_i$ is scattered into direction $\hat{s}$). Subscript $\eta{}$ represents wavenumber (the inverse of wavelength) and $i$ marks an incident direction. This equation describes how the intensity of a pencil of rays changes along a path length $\hat{\textbf{s}}$. A ray intensity will decrease due to absorption ($\kappa{}_\eta{}I_\eta{}$, also known as Beer's law) and out-scattering ($\sigma{}_{s\eta{}}I_\eta{}$). And the ray intensity will increase due to further emission along the path-length ($\kappa{}_\eta{}I_{b\eta{}}$) and scattering of rays from other directions into the path length of interest ($\frac{\sigma{}_s}{4\pi}\int_{4\pi{}}{I_\eta{}(\hat{s})\Phi_\eta{}(\hat{s}_i,\hat{s})}\,d\Omega{}_i$). 

The existence of both an integral and a differential in \ref{eq:RTE2} poses a complexity problem as the equation being modeled is now an integro-differential equation. 
This requires both modeling of local phenomena and the influence of global phenomena on local conditions. 
This all-to-all nature is the reason for the immense computational difficulty of modeling the RTE. 
In addition, as is explained in section \ref{Sec:Nongray}, the highly intermittent nature of the emission spectrum of gas imposes additional modeling difficulty.

Absorption coefficient, $\kappa_{\eta{}}$ and scattering coefficient $\sigma{}_\eta{}$ are often combined into the extinction coefficient $\beta{}_\eta{}$. Equation \ref{eq:RTE2} can be rewritten for greater conceptual understanding to \ref{eq:RTErewritten}.

\begin{equation}
    \frac{dI_\eta{}}{ds} = \beta{}_\eta{}[S(s',\hat{s})-I_\eta{}]
    \label{eq:RTErewritten}
\end{equation}

\begin{equation}
    S(s',\hat{s}) = (1-\omega{}_\eta{})I_{b\eta{}}-\frac{\omega{}_\eta{}}{4\pi}\int_{4\pi{}}{I_\eta{}(\hat{s})\Phi_\eta{}(\hat{s}_i,\hat{s})}\,d\Omega{}_i
    \label{eq:RTE_S}
\end{equation}

Where $S_\eta{}(s',\hat{s})$ is the ray-intensity source term eq. \ref{eq:RTE_S}, and $\omega{}_\eta{}=\sigma{}_{s\eta{}}/\beta_{\eta{}}$ is the single-scattering albedo. Eq. \ref{eq:RTErewritten} shows that the ray intensity increases due to some source term, and decreases proportial to it's current intensity.

\subsection{Non-gray effects}\label{Sec:Nongray}
Like all substances, gases emit radiation along the electromagnetic spectrum. Under most conditions for combustion phenomena, radiative emission is isolated to the infrared part of the spectrum, between wavelengths of 0.7 to 14 $\mu{}$m.

The black body spectrum in section \ref{Sec:FundOfRad} can approximate the emission spectrum of many solid materials, but is inadvisable to be used for gas emission. 
Most solid structures have a multitude of modes of oscillations, from the crystal-lattice structure down to the molecular energy states. As a result, the emission spectrum is continuous and can be approximated as a constant fraction of the black body emission spectrum (gray) \cite{Howell2010ThermalTransfer}. 
Gases, however, are restricted to the oscillation modes from the natural energy states of the molecules in the mixture. Quantum mechanics postulates that these energy states are discrete, and therefore the change in energy states and resulting emission frequencies must also be discrete.


\textbf{INSERT EMISSION SPECTRUM HERE}

Figure above shows the emission spectrum from a flame. The largest contributors to emission in most hydrocarbon flames are included: $CO_2$, $CO$, $H_2O$, and soot.
\textit{Ab initio} modeling and empirical evidence can be used to obtain approximations for the emission spectra of a molecule at a specified pressure and temperature. This information can be tabulated to provide a radiation solver the ability to determine the absorption coefficient at a location in the combustion mixture, for a given intersecting ray wavelength. Various methods to model non-gray effects are introduced in Ch. \ref{chapter:Modeling}.

\subsection{Effect of radiation on fluid flow}
Combustion sees a redistribution of thermal energy as a result of thermal radiation. 
Neglecting thermal radiation may lead to large over-predictions of flow temperature by upwards of 200K \cite{Tur}

% =============================================================== %
\section{Soot models}

\subsubsection{Soot model expectations}

\subsubsection{Difficulties in soot modeling}





\subsection{Classification of soot models}

\subsubsection{Empirical soot models}


\subsubsection{Semi-empirical soot models}



\subsubsection{Detailed soot models}



\subsection{Review of soot research}



\subsection{Soot models used in the present work} \label{2Eq_sootModel}

\subsubsection{Two-equation model} 
  
  
  
% =============================================================== %
\section{Soot modeling sensitivity}

\subsection{Chemical mechanisms}


\subsection{Precursor species}



\subsection{Reaction rates}
