\section{Pratt \& Whitney Aeronautical Combustor}
A three dimensional Pratt \& Whitney combustor \textbf{(Block C)}? geometry was next tested for verification and profiling of the present radiation model. Solver profiling is presented alongside general observations regarding the effects of radiation within the system.
The combustor geometry follows a Rich-Quench-Lean (RQL) configuration, an optimal design for both emissions and stability.
A single time-step from an LES of the combustor was taken for a frozen-field analysis. 
The combustor geometry is constructed to for highly turbulent flow with a high degree of separation to allow for a sufficiently stabilized flame.
A re-circulation zone immediately behind the inlet section induces a swirl which facilitates mixing of the reactants, and a downstream dilution hole results in a complex chemical mixing process for the remaining reactions to occur in a fuel-lean condition. This results in a more complex profile of radiation throughout the fluid and the walls.
The resulting influences of radiation within the fluid and along the walls are non-monotonic with respect to distance from the inlet, and especially with consideration of the influences of non-gray emission and absorption. 
All presented parameters are normalized by median values obtained through the computational cells or faces.


\subsection{Case setup}
The combustor geometry consists of one slice of the annular combustor. The domain is meshed into a grid of 16,373,876 hexahedral cells of varying shapes and sizes. 
Radiation is modeled with adaptive emission, as discussed in section \ref{section:SmallPoolFlame}, and specularly reflective boundary conditions are imposed on the periodic boundaries.
The effects of soot, spectral emission patterns, and wall-incident radiation were all captured within the model, and detailed account of these effects alongside profiling of the various parallel implementations of the model are presented as demonstration.
Contributions from the three primary emitting and absorbing chemical species are accounted for: CO$_2$, H$_2$O, and CO, as well as the influences of soot are also accounted for.
The effects of liquid fuel spray on ray-scattering and absorption and all turbulence-radiation interactions are neglected.

\subsection{Results}

\subsubsection{Radiative Emission}
It is of interest to first investigate the radiative emission characteristics of the combustor.
Equation \ref{eq:RadEmis} (repeated in volumetric form in eq. \ref{eq:RadEmis2}) defines the influence of both Planck mean absorption coefficient $\kappa{}$ and temperature $T$ on volumetric radiative emission.
\begin{equation}
    Q_{emi}=4\kappa{}_{p}n^2\sigma{}T^4
    \label{eq:RadEmis2}
\end{equation}
Where the refractive index $n$ is assumed constant. 
The volumetric emission is a function of both Planck mean absorption coefficient and temperature. It is important to note, however, that Planck mean absorption coefficient is itself a function of temperature, in addition to the chemical composition of the mixture. 

\textbf{Figure: 3 Scatter plots of emission, kplanck, and fvSoot vs temperature}

Parameters from computational cell data are displayed in normalized form in fig. \ref{fig:PW_Scatterplot_emission}. 
Suprisingly, radiative emission displays a non-monotonic trend with respect to temperature. The scaling reflects the expected fourth-order nature before the normalized temperature value of $0.7$, but decreases sharply as temperature continues to increase.
At this point, the contribution of temperature to radiative emission is superseded by the decrease in Planck-mean absorption coefficient.
This is evident in figure \ref{fig:PW_kplanck_scatter}.
Between temperatures of T=$0.4$ and $0.7$, $\kappa{}_p$ shows a gradual decline. After a value of T=$0.7$, however, $\kappa{}_{p}$ begins to decrease much more rapidly towards $0$. 
The various contributions of the $\kappa{}$ are presented in eq. \ref{eq:Kplanck_definition}.
\begin{equation}
    \kappa{}_p=f_{v,soot}\kappa{}_{soot}+X_{CO_2}\kappa{}_{CO_2}+X_{H_2O}\kappa{}_{H_2O}+X_{CO}\kappa{}_{CO}
    \label{eq:Kplanck_definition}
\end{equation}
Where $f_{v,soot}$ is the soot volume fraction, and $X_i$ is the mole fraction of species $i$.
The contributions of each of the terms in equation \ref{eq:Kplanck_definition} are presented in fig. \ref{fig:PW_kplanck_contributions}. All values are normalized by the contributions of soot.
It is evident that soot plays the most significant role in defining the absorption coefficient at present conditions. Additionally, the sharp decrease of Planck mean absorption coefficient present in fig. \ref{fig:PW_Scatterplot_kplanck} at T=$0.7$, also presents itself in the temperature profile of soot volume fraction in fig. \ref{fig:PW_Scatterplot_soot}.
This sudden drop represents a crossover point. At this temperature, soot particles are consumed rapidly by the gas, and the resulting radiative emission rapidly decreases.

\subsubsection{Radiative Absorption}
As mentioned in chapter \ref{chapter:Importance}, the redistribution of thermal energy from radiation is significant for multiple reasons: radiation can produce changes in the fluid dynamics, turbulence, wall heat flux, and the chemical production of pollutants.
The Pratt \& Whitney combustor displays a relevant degree involvement of radiative absorption to each of these characteristics.

Figure \ref{fig:PW_WallAverages} presents the averaged wall heat flux for the inner diameter (ID) and outer diameter (OD) surfaces of the annular combustor slice. 
The walls are assumed to be cold and black; they absorb all incident radiation and re-emit a negligible amount. 
Wall-incident radiation increases throughout the rich-burn section of the combustor, reaches a local maximum just before the dilution holes, and decreases throughout the lean-burn section.
This corroborates the results presented by \citet{Gamil2020AssessmentChamber}, where the fuel-rich section of the flame was shown to emit highly and increase the temperatures of the combustor walls.

\textbf{Figure: Average wall heat flux profile as a function of distance from the inlet}

To analyze the contribution of non-gray absorption to the flame, gray and non-gray wall heat flux is presented in fig. \ref{fig:WallHeatFlux_GrayVsNongray}.
Both ID and OD walls see up to 50 to 60\% decreases in wall heat flux when the line-by-line radiation model is introduced. Decreases diminish slightly in the fuel-rich part of the combustor to 30 to 40\%, but remain substantial.
Net percent differences in wall heating is presented for various boundaries in fig. \ref{table:PW_GrayVsNonGray}.
The decrease in radiative transmission to the walls is a direct result of the increased re-absorption within the flame. 
The introduction of non-gray modeling causes increased attenuation of radiation. Changes in radiative re-absorptions are presented for both soot and 


\textbf{$\times$ Figure: gray vs non-gray averaged heat flux values in the fluid}

\begin{table}[h!]
\centering
\begin{tabular}{||c c||} 
 \hline
 Boundary & Percent Decrease \\ [0.5ex] 
 \hline\hline
 OD Wall & 6 \\ 
 ID Wall & 7  \\
 Turbine & 545  \\
 Inlet & 545  \\ [1ex] 
 \hline
\end{tabular}
\caption{Percent differences in incident radiative power between gray and non-gray radiation models.}
\label{table:PW_GrayVsNonGray}
\end{table}

The overall budgeted contributions to wall heat flux are diagrammed in fig. \ref{fig:PW_emission_contributions} across the emission spectrum.
The intermittent nature of the emission from the molecules is reflected by the accounted contributions of CO$_2$, H$_2$O, and CO. The emission of soot is more continuous throughout the spectrum, as expected.
Overall, soot contributes the majority of emission from the flame, with emission occurring generally at more moderate temperatures, as previously discussed.

\textbf{Figure:Budgeted contributions of species and soot across the spectrum.}

\subsection{Profiling}
The new MCRT implementation displayed a significant speedup compared to the established fortran model. A comparison of runtimes is present in fig. \ref{table:PW_runtime_table}

\begin{table}[h!]
\centering
\begin{tabular}{||c c c c||} 
 \hline
 Variation & MCRT-Fort & MCRT-Kokkos & MCRT-ArborX \\ [0.5ex] 
 \hline\hline
 Gray & 5770s     & 2308 & 57  \\ % TODO
 Non-Gray & 5770s & 3000 & 78 \\ [1ex] 
 \hline
\end{tabular}
\caption{Runtime comparisons for the Pratt \& Whitney combustor geometry.}
\label{table:PW_runtime_table}
\end{table}


- Runtime gray

- Runtime non-gray