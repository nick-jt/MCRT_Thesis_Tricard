\addchapheadtotoc

\chapter{Introduction} \label{chapter:Introduction}
% Combustion has long been an effective means of energy conversion. Transportation systems, building heating and cooling, and power for electrical grids are just a few examples of its many applications. Over 82\% of energy produced on earth is produced through combustion~\cite{2021BpEnergy}. 
% At the same time, the explosive nature of the process may also result in its unwanted presence (e.g. rocket explosions, forest fires, house-fires, detonative combustion).
% Combustion is therefore a process that both can be used and must be managed. 
% Meanwhile the ongoing focus to reduce global emissions also imposes restraints on combustion researchers and scientists. Pollutants such as CO${}_2$, nitrous oxides, and soot have been shown to have a negative influence on our environment.
% So while the energy storage and release mechanisms that accompany combustion prove convenient for a lot of systems, they simultaneously presents a challenge for engineers and researchers.
% The need for improvement management of thermal loads, greater interconnected transport systems, development of heat-resistant materials, and an ongoing push to reduce global emissions must be met with an enhanced understanding of these physical systems.
% This enhanced understanding requires thorough understanding of the complex fluid dynamics, heat transfer, and chemistry that accompany combustion.
% Radiation is one of the most prominent physical processes that occurs during... 

\section{Background/Motivation}
Radiative heat transfer in combustion systems has attracted increasing attention in recent years.
For example, aeronautical engines tend to be designed to operate at higher temperature and pressure, in order to improve the overall thermal efficiency.
%high temperatures of combustion products may lead to high degrees of radiative emission.
Consequently, the higher radiative loads will redistribute thermal energy throughout the combustor, and result in higher temperatures and radiative flux in the surrounding casing, changes in chemistry dynamics such as flame extinction and pollutant emissions, and, thermodynamically, a net energy loss from the system which can no longer contribute to any desired work output.
Likewise, as wildland fires become more frequent due to climate change, radiation from them can contribute extensively to the widespread propagation and expansiveness of a flame, amounting to between 10\% and 90\% of the total energy flux~\cite{Valendik2008EffectEnvironment}. 
The emission from the fire is known to heat nearby combustible materials, dry them substantially, and accelerate flame propagation. At the same time, nearby firemen, who are oftentimes only meters away from the flame, may be exposed to dangerous radiative loads up to 12 kW/m$^2$~\cite{Valendik2008EffectEnvironment} which can instantaneously burn to any unprotected skin.

Radiation in combustion systems may potentially impact the fluid motion, turbulence, heat flux to surrounding objects, and chemistry dynamics. Specifically, several effects may be expected regarding the influence of radiation in combustion systems, as discussed in reviews from Modest and Haworth~\cite{Modest2016RadiativeSystems}, Coehlo~\cite{Coelho2018RadiativeSystems}, and Liu et. al.~\cite{Liu2020TheFlames}:
\begin{itemize}
    \item Radiation contributes thermal energy at the same order of magnitude as convection to the boundaries of many enclosed combustion systems~\cite{Gamil2020AssessmentChamber,Johnson2021AnalysisMethod}
    \item Radiation has importance in pollutant formations (NO${}_x$, soot, CO)~\cite{Ihme2008ModelingFormulation,Habibi2007TurbulenceFlames}
    \item Sooting flames may have more pronounced thermal radiation characteristics
\end{itemize}
Regarding modeling of radiation in combustion system:
\begin{itemize}
    \item Neglecting radiative emission results in temperature overprediction~\cite{Gamil2020AssessmentChamber}
    \item Neglecting radiative absorption results in temperature underprediction
    \item Accounting for turbulence-radiation interaction (TRI) increases radiative emission and reduces temperatures
    \item Non-gray modeling of radiation will increase radiative re-absorption compared to gray modeling~\cite{Wu2021LimitationsFires}
\end{itemize}


%Thermal radiation is often the most dominant form of heat transfer that accompanies many combustion systems. The high temperatures from the rapid energy release combined with heavily radiating combustion products result in a 

%Therefore, 
Misrepresentation of radiation in combustion modeling can lead to significant errors in the prediction of flame properties. These errors may present themselves, for example, in $200$~K differences of peak flame temperature in gas-turbines~\cite{Gamil2020AssessmentChamber}, $50$\% under-predictions of heat loss from a piston engine~\cite{Modest2016RadiativeSystems}, or significant mispredictions of the damaging effects and the rate of spread of forest fires~\cite{Valendik2008EffectEnvironment}. 
The modeling procedure of radiation in a participating medium, however, is complex and computationally expensive, as emission and absorption between every point within the fluid system must be accounted for. Consequently, despite its significant role, many scientists and engineers apply simplified models of radiation such as “optically thin” and “gray” models, which may provide misleading and/or incomplete information as to the contribution of radiation. 
A significant objective in the combustion and radiation communities is therefore to develop faster radiation modeling approaches that can still accurately account for its non-linear, non-local and coupled influences to combustion.

%\section{Influence of radiation from combustion processes}

%Oftentimes, radiation can present a damaging effect to surrounding objects.





These influences drive scientists and engineers to produce faster and more accurate radiation models to couple to their combustion simulations.


% Many computational models have been proposed for radiation, oftentimes accompanying Computational Fluid Dynamics (CFD) solutions to model the fluid dynamics and chemistry. 
% The most accurate approach, Monte-Carlo Ray Tracing (MCRT), requires a significant amount of computational resources. As a result, scientists and engineers have historically resorted to lower-fidelity radiation models at the cost of increased uncertainty.
% Over time, however, new implementations have emerged accelerating MCRT, largely driven by enhancements in computational hardware and software. 
% Scientists and engineers must continuously adapt their models to these new approaches in order to maintain the best available understanding of the influences of radiation in their studies. 




% Conversely, thermal radiation from combustion also provides meaningful contribution, oftentimes in a ways that are apparent in day to day life.
% The origins of the human usage of fire are believed to result from the radiative warmth that fire provides~\cite{Scott2018WhenDiscover,Baird2015WhenMe}.



% Thermal radiation plays an integral role in the recent advancements of combustion technologies.




% In its most ideal form, radiative emission energy takes the form of eq. \ref{eq:RadEnergy}.
% \begin{equation}
%     E_{rad}=\epsilon{}\sigma{}T^4
%     \label{eq:RadEnergy}
% \end{equation}
% Where E$_{rad}$ is the radiative emissive power in W/m$^3$, $\epsilon{}$ is the dimensionless emissivity (or emittance), $\sigma{}$ is the Stefan-Boltzmann constant in W/m$^2$K$^4$, and T is the temperature in Kelvin. 
% The fourth power relationship of emissive power with temperature represents a significant reason why radiation is often more prominent in combustion, where temperatures can reach upwards of $2000$ K~\cite{Modest2016RadiativeSystems}.
% The high heat fluxes can influence becomes important for pollutant formation, flame speeds, and threshold phenomena (near-limits)~\cite{Modest2016RadiativeSystems,Coelho2018RadiativeSystems,Liu2020TheFlames}. Including radiation can have an effect on the fluid dynamics, turbulence, boundaries, and mixture properties of a combustion simulation.

% Many turbulent flows are modeled using Reynolds-Averaged Navier-Stokes (RANS) and Large Eddy Simulations (LES), both of which rely on the averaging of the governing equations of combustion. This results in the need for models describing the unresolved fluctuations of radiative properties. The influence of unresolved scales on radiation is known as turbulence-radiation interaction (TRI). Emission-TRI and absorption-TRI are two radiative phenomena which can complicate radiation modeling, and have been extensively studied.


\section{Modeling challenge and approach}
The immense computational expense required for integration of the multitude of coupled equations within a combustion simulation are computationally prohibitive beyond laboratory-scale flames, even with modern computing resources. Consequently, thermal radiation is either ignored or modeled using oversimplified approaches because it adds significantly to this expense.
In attempt to maximize accuracy at the limitations of present resources, a number of models have been introduced to model thermal radiation while minimizing computational burden.
Of those, many rely on complex mathematical assumptions and simplifications which can be difficult to learn, implement, and may lead to inaccurate results for an unfamiliar user. 


The Monte-Carlo Ray Tracing (MCRT) method stands out as the most accurate, robust, and conceptually simple of the available models~\cite{Tesse2002RadiativeApproach,Modest2013RadiativeTransfer,Coelho2018RadiativeSystems}, and is believed by many to be the best radiation model as computational power increases~\cite{Howell2010ThermalTransfer}.
In MCRT, random rays are initialized and are traced from their source location to other locations in the domain, with the intention of stochastically approximating transport of radiation throughout the geometry.
The resulting process can be intuitively thought of as the tracing of `photon packets' traveling through the domain.
As a result of this approach, MCRT can account for many of the same effects that electromagnetic rays undergo during their travel.

Although MCRT is often considered the most accurate, it has also been viewed traditionally as the most computationally expensive.
This additional cost results from the stochastic nature of MCRT. A large sample size of rays must be emitted and traced to obtain a statistically meaningful solution. 
Too few rays may result in excessive noise, while too many will require a longer runtime.
Much effort has been put into making the computation more manageable while maintaining the benefits MCRT has to offer over other methods~\cite{Liu2020TheFlames,Tesse2002RadiativeApproach,Zeeb2001AnGeometries,Modest2003BackwardTransfer,Howell2010ThermalTransfer}.

In this paper, we implement an accelerated implementation of MCRT that is coupled with the \verb|OpenFOAM| open-sourced CFD platform~\cite{Weller1998ATechniques}. 
This implementation relies on Kokkos programming model for performance-portable parallel computing with interchangeable parallel computing back-ends~\cite{Trott2021KokkosEra}. This enables the use of Graphics Processing Units (GPUs) to dramatically accelerate the ray-tracing procedure, a process has been done very few times, but has shown significant speedups~\cite{Silvestri2019ASimulation,Humphrey2016RadiativeRefinement,Heymann2012GPU-basedAGN}. 
The solver is extended to distributed-memory computing platforms through the use of the Message Passing Interface (MPI).
Additionally, the ArborX geometric search library is introduced to implement a bounding-volume hierarchy based structure which has shown efficiency boosts in the field of graphics processing~\cite{Lebrun-Grandie2019ArborX:Library}. 

\section{Organization}
This thesis will first present background on the fundamentals of radiation and its importance in combustion in chapter \ref{chapter:Importance}. 
Then, various methods of modeling radiation will be presented in chapter \ref{chapter:Modeling}.
This chapter will include an overview of the present MCRT approach, the supplemental libraries used, and a description of coupling to \verb|OpenFOAM|.
Chapter \ref{chapter:Example} will provide a demonstration of the model on both canonical and practical configurations, including a plane-parallel medium, a backward-facing step, turbulent pool fire, and a Pratt \& Whitney combustor. Various observations will be made regarding the influence of radiation and computational speedup will be presented in this chapter.
Finally, conclusions and future work will be discussed in chapter \ref{chapter:conclusion}.